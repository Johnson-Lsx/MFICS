\documentclass[12pt]{article}
\usepackage{amsmath}
\usepackage{amssymb}
\usepackage{amsthm}
\usepackage{enumerate}
\usepackage{hyperref}
\usepackage{txfonts}
\usepackage{amsmath}
\usepackage{amssymb}
\usepackage{amscd}
\usepackage{amsmath, mathtools,amssymb}
\usepackage{amsfonts,semantic,colortbl,mathrsfs,stmaryrd}
\usepackage{enumerate}
\usepackage{multirow}
\usepackage{graphicx}
\date{Feb 14, 2012}
\newtheorem{thm}{Theorem}
\newtheorem{lemma}[thm]{Lemma}
\newtheorem{fact}[thm]{Fact}
\newtheorem{cor}[thm]{Corollary}
\newtheorem{eg}{Example}
\newtheorem{hw}{Problem}
\newcommand{\xor}{\otimes}
\newenvironment{sol}
  {\par\vspace{3mm}\noindent{\it Solution}.}
  {\qed}
\begin{document}
\begin{center}
{\LARGE\bf Homework 4}\\
\vspace{2mm}
\end{center}


\begin{hw}
  \begin{enumerate}
    \item  Determine the coefficient of $x^{50}$ in $(x^7+x^8+x^9+x^{10}+\cdots)^6$
    \item  Determine the coefficient of $x^3$ in $(2+x)^{\frac{3}{2}}/(1-x)$
    \item  Determine the coefficient of $x^4$ in $(2+3x)^{5}\sqrt{1-x}$
  \end{enumerate}
\end{hw}
\begin{sol}
	\begin{enumerate}
		\item We know that $x^7 + x^8 + x^9 + x^{10} + \cdots$ is the generating function of the sequence $(0,0,0,0,0,0,0,1,1,1,\cdots)$. We can get the closed form of this generating function, which is $\frac{x^7}{1-x}$. Thus, the original expression can be rewritten as $\left(\frac{x^7}{1-x}\right)^6 = \frac{x^{42}}{(1-x)^6}$. Since $(1-x)^{-6} = {5 \choose 5} + {6 \choose 5}x + {7 \choose 5}x^2 + \cdots + {5+k \choose 5}x^k + \cdots$, the coefficient of $x^{50}$ is ${13 \choose 5}$.
		
		\item We can rewrite the expression as $\sum_{k=0}^{\infty}{3/2 \choose k}2^{3/2 - k}x^k(1+x+x^2+\cdots)$, so the coefficient of $x^3$ is $\sum_{k=0}^{3}{3/2 \choose k}2^{3/2 - k} = 2^{3/2} +\frac{3}{2}\times2^{1/2} + \frac{3}{8} \times 2^{-1/2} - \frac{1}{16}\times 2^{-3/2}$
		
		\item We can rewrite the expression as $\sum_{i=0}^{\infty}{5 \choose i}2^{5-i}(3x)^i\cdot\sum_{j=0}^{\infty}{1/2 \choose j}(-x)^j$, the the coefficient of $x^4$ is $\sum_{k=0}^{4}{5 \choose k}2^{5-k}3^{k}{1/2 \choose 4-k}(-1)^{4-k}$
	\end{enumerate}
\end{sol}


\begin{hw} Find generating functions for the following sequences (express them in a closed form, without infinite series!):
\begin{enumerate}
  \item $0,0,0,0,-6,6,-6,6,-6,\cdots$.
  \item $1,0,1,0,1,0,\cdots.$
  \item $1,2,1,4,1,8\cdots$
\end{enumerate}
\end{hw}
\begin{sol}
	\begin{enumerate}
		\item The generating function of $(1,1,1,\cdots)$ is $1 + x + x^2 + \cdots$, and the closed form of the generating function is $\frac{1}{1-x}$, so the closed form of the generating function of $(1,-1,1,-1,\cdots)$ is $\frac{1}{1+x}$. The generating function of $(-6,6,-6,6,\cdots)$ is $\frac{-6}{1+x}$. Thus the generating function of $(0,0,0,0,-6,6,-6,6,-6,\cdots)$ is $\frac{-6x^4}{1+x}$.
		
		\item The generating function of $(1,1,1,1,\cdots)$ is $1 + x + x^2 + x^3 + \cdots$ and the generating function of $(1,-1,1,-1,\cdots)$ is $1 - x + x^2 - x^3 + \cdots$. Thus, the generating function of $(1,0,1,0,1,0,\cdots)$ is $\frac{1}{2}(\frac{1}{1-x}+\frac{1}{1+x}) = \frac{1}{1-x^2}$.
		
		\item The generating function of $(1,2,4,8,\cdots)$ is $\frac{1}{1-2x}$, so the generating function of $(1,0,2,0,4,0,8,\cdots)$ is $\frac{1}{1-2x^2}$. Since the generating function of $(1,0,1,0,\cdots)$ is $\frac{1}{1-x^2}$, so the generating function of $(0,1,0,1,\cdots)$ is $\frac{x}{1-x^2}$. Thus, the generating function of $(1,1,2,1,4,1,8,\cdots)$ is $\frac{x}{1-x^2} + \frac{1}{1-2x^2}$, finally we have the generating function of $(1,2,1,4,1,8,\cdots)$ is $\left(\frac{x}{1-x^2} + \frac{1}{1-2x^2} - 1\right)/x = -\frac{2x^3+2x^2-2X-1}{(1-x^2)(1-2x^2)}$.
	\end{enumerate}
\end{sol}


\begin{hw}
Let $a_n$ be the number of ordered triples $\langle i,j,k\rangle$ of integer numbers such that $i\geq 0,j\geq 1, k\geq 1$, and $i+3j+3k=n$. Find the generating function of the sequence $(a_0, a_1, a_2, \ldots)$ and calculate a formula for $a_n$.
\end{hw}
\begin{sol}
	The value of $a_n$ is equal to the coefficient of $x^n$ in the result of $(1+x+x^2+x^3+\cdots)(x^3+x^6+x^9+\cdots)(x^3+x^6+x^9+\cdots)$. Thus, the generating function of $a_n$ is $$(1+x+x^2+x^3+\cdots)(x^3+x^6+x^9+\cdots)(x^3+x^6+x^9+\cdots)$$
	Since we can rewrite the expression as $\frac{1}{1-x}\cdot\frac{x^3}{1-x^3}\cdot\frac{x^3}{1-x^3}$, a formula for $a_n$ can be 
	\begin{displaymath}
		a_n = \left\{ 
		\begin{array}{ll}
			0 & n < 6 \\
			(-1)^{\frac{n-6}{3}}{n/3 \choose 2}  & n \geq 6 \ \textrm{and} \ n \equiv 0 \ (\textrm{mod} 3) \\
			(-1)^{\frac{n-7}{3}}{(n-1)/3 \choose 2}  & n \geq 6 \ \textrm{and} \ n \equiv 1 \ (\textrm{mod} 3) \\
			(-1)^{\frac{n-8}{3}}{(n-2)/3 \choose 2}  & n \geq 6 \ \textrm{and} \ n \equiv 2 \ (\textrm{mod} 3) \\
		\end{array}
		\right.
	\end{displaymath}
\end{sol}


\begin{hw}
Express the $n^{th}$ term of the sequences given by the following recurrence relations



\begin{enumerate}
 \item $a_0=2, a_1=3, a_{n+2}=3a_n - 2a_{n+1} $ $(n=0,1,2,\ldots)$.
 \item $a_0=1, a_{n+1}=2a_n+3$ $(n=0,1,2,\ldots).$
\end{enumerate}
\end{hw}
\begin{sol}
	\begin{enumerate}
		\item The characteristic function is $x^2 + 2x -3 = 0$, there are two different solutions, which are $x_1 = -3, x_2 = 1$. Thus, we have $a_n = c_1(-3)^n + c_2(1)^n$. Since $a_0 = 2,a_1 = 3$, we can figure out taht $c_1 = -1/4,c_2 = 9/4$, so $a_n = -\frac{1}{4}(-3)^n + \frac{9}{4}$.
		
		\item The homogeneous part is $x = 2$, to find one specific solution for the recurrence relation, we try $a_n = p2^n + s$, then we have $p2^{n+1} + s = 2\cdot p2^n + 2s + 3$, so $s = -3$. Since $a_0 = 1$ we have $(c_1 + p)2^0 - 3 = 1$, so $(c_1+p) = 4$. Finally, we have $a_n = 2^{n+2} - 3$.
	\end{enumerate}
\end{sol}


\begin{hw}
Solve the recurrence relation $a_{n+2}=\sqrt{a_{n+1}a_n}$ with initial conditions $a_0=2, a_1=8$ and find $\lim_{n\rightarrow \infty}a_n $.
\end{hw}
\begin{sol}
	Let $b_n = \log_{2}a_n$, then we have $2b_{n+2} = b_{n+1} + b_{n}$, from this recurrence relation we can figure out that $b_{n} = -\frac{4}{3}(-\frac{1}{2})^n + \frac{7}{3}$. Thus, we have $a_n = 2^{-\frac{4}{3}(-\frac{1}{2})^n + \frac{7}{3}}$ and $\lim_{n\rightarrow \infty}a_n = 2^{\frac{7}{3}}$.
\end{sol}




\end{document}

%%% Local Variables:
%%% mode: tex-pdf
%%% TeX-master: t
%%% End:
