\documentclass[12pt]{article}
\usepackage{amsmath}
\usepackage{amssymb}
\usepackage{amsthm}
\usepackage{enumerate}
\usepackage{hyperref}
\usepackage{xypic}
\usepackage{txfonts}
\usepackage{amsmath}
\usepackage{amssymb}
\usepackage{amscd}
\usepackage{amsmath, mathtools,amssymb}
\usepackage{amsfonts,semantic,colortbl,mathrsfs,stmaryrd}
\usepackage{enumerate}
\usepackage{multirow}
\usepackage{graphicx}
\date{Feb 14, 2012}
\newtheorem{thm}{Theorem}
\newtheorem{lemma}[thm]{Lemma}
\newtheorem{fact}[thm]{Fact}
\newtheorem{cor}[thm]{Corollary}
\newtheorem{eg}{Example}
\newtheorem{hw}{Problem}
\newcommand{\xor}{\otimes}
\newenvironment{sol}
  {\par\vspace{3mm}\noindent{\it Solution}.}
  {\qed}
\begin{document}
\begin{center}
{\LARGE\bf Homework 8}\\
\vspace{2mm}
\end{center}

\begin{hw}
We have 27 fair coins and one counterfeit coin (28 coins in all), which looks like a fair coin but is a bit heavier. Show that one needs at least 4 weighings to determine the counterfeit coin. We have no calibrated weights, and in one weighing we can only find out which of two groups of some $k$ coins each is heavier, assuming that if both groups consist of fair coins only the result is an equilibrium.
\end{hw}
\begin{sol}
	We first say that there is one way to determine the counterfeit coin using $4$ weighings. We can do it in the following way.
	\begin{enumerate}
		\item Divide the total $28$ coins into $2$ parts, each part consists of $14$ coins. We then weigh the two parts and choose the heavier part.
		
		\item Divide the $14$ coins into $2$ parts, each part consists of $7$ coins. We then weigh the two parts and choose the heavier part.
		
		\item Divide the $7$ coins into $3$ parts, two of them consist of $3$ coins and the rest one consists of only $1$ coin. We then weigh the two parts which consist of $3$ coins, if they are the same weight, then the rest one is the counterfeit coin, if not, we choose the heavier part.
		
		\item  Divide the $3$ coins into $3$ parts, each consists of $1$ coin. We randomly choose two parts to weigh, and if they are the same weight, then the rest one is the counterfeit coin, if not, the heavier one is the counterfeit coin.
	\end{enumerate}
	We can also see from the above process that if we only use $3$ weighings, we may fail to determine the counterfeit coin. Thus, we need at least $4$ weighings to determine the counterfeit coin.
\end{sol}


\begin{hw}
\begin{enumerate}
  \item Prove that, for every integer $n$, there exists a coloring of the edges of the complete graph $K_n$ by two colors so that the total number of monochromatic copies of $K_4$ is at most ${n\choose 4}2^{-5}$.
  \item Give a randomized algorithm for finding a coloring with at most ${n \choose 4}2^{-5}$ monochromatic (i.e. single-color) copies of $K_4$ that runs in expected time polynomial in $n$.
\end{enumerate}
\end{hw}
\begin{sol}
	\begin{enumerate}
		\item For each edge in the complete graph $K_n$, we randomly color it red or blue. Thus, the probability that a $K_4$ subgraph is monochromatic equals to $2 \cdot \frac{1}{2^{{4 \choose 2}}} = 2^{-5}$. The expectation of the number of monochromatic copies of $K_4$ is ${n \choose 4}2^{-5}$. Thus, there must be a coloring of the edges of the complete graph $K_n$ by two colors so that the total number of monochromatic copies of $K_4$ is at most ${n\choose 4}2^{-5}$.
		
		\item The randomized algorithm is coloring the edges of $K_n$ randomly into red or blue and find if there is a coloring meets the requirement. We say that the algorithm runs in expected time polynomial in $n$. Let $p = Pr(X \leq {n \choose 4}2^{-5})$, where $X$ is the number of $K_4$ subgraph in $K_n$. From question $1$, we have
		\begin{displaymath}
			\begin{array}{lcl}
				{n \choose 4}2^{-5} & = & E(X) \\
									& = & \sum_{i \leq {n \choose 4}2^{-5}}i \cdot Pr(X = i) + \sum_{i > {n \choose 4}2^{-5}}i \cdot Pr(X = i) \\
									& \geq & p + (1-p)({n \choose 4} + 1)
			\end{array}
		\end{displaymath}
		Thus, $p \geq \frac{32}{{n \choose 4}}$. The excepted number of samples before finding a coloring is therefore just $\frac{{n \choose 4}}{32}$ and each sample costs at most $O({n \choose 4})$ time. Thus, the algorithm runs in expected time polynomial in $n$.
	\end{enumerate}
\end{sol}



\begin{hw}
Use the Lovasz local lemma to show that if \[4 {k \choose 2} {n \choose {k-2}}2^{1-{k\choose 2}}\leq 1\]
then it is possible to color the edges of $K_n$ with two colors so that it has no monochromatic (i.e. single color) $K_k$ subgraph.
\end{hw}
\begin{sol}
	Let $E_{i}$ denote that the $i$th $K_k$ subgraph is monochromatic, we have $Pr(E_i) = 2 \cdot \frac{1}{2^{{k \choose 2}}} = 2^{1-{k\choose 2}}$. And the degree of the dependency graph given by $E_1, E_2, \dots, E_n$ is bounded by ${k \choose 2} {n \choose {k-2}}$. From the question we have $4 {k \choose 2} {n \choose {k-2}}2^{1-{k\choose 2}}\leq 1$. Thus, by \emph{Lovasz local lemma} we have $Pr(\bigcap_{i = 1}^n \overline{E_i}) > 0$, that is, it is possible to color the edges of $K_n$ with two colors so that it has no monochromatic $K_k$ subgraph.
\end{sol}


\end{document}

%%% Local Variables:
%%% mode: tex-pdf
%%% TeX-master: t
%%% End:
