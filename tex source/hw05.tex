\documentclass[12pt]{article}
\usepackage{amsmath}
\usepackage{amssymb}
\usepackage{amsthm}
\usepackage{enumerate}
\usepackage{xypic}
\usepackage{txfonts}
\usepackage{amsmath}
\usepackage{amssymb}
\usepackage{amscd}
\usepackage{amsmath, mathtools,amssymb}
\usepackage{amsfonts,semantic,colortbl,mathrsfs,stmaryrd}
\usepackage{enumerate}
\usepackage{multirow}
\usepackage{graphicx}
\usepackage{hyperref}
\hypersetup{colorlinks=true,linkcolor=black}
\date{Feb 14, 2012}
\newtheorem{thm}{Theorem}
\newtheorem{lemma}[thm]{Lemma}
\newtheorem{fact}[thm]{Fact}
\newtheorem{cor}[thm]{Corollary}
\newtheorem{eg}{Example}
\newtheorem{hw}{Problem}
\newcommand{\xor}{\otimes}
\newenvironment{sol}
  {\par\vspace{3mm}\noindent{\it Solution}.}
  {\qed}
\begin{document}
\begin{center}
{\LARGE\bf Homework 5}\\
\vspace{2mm}
\end{center}


\begin{hw}Fill in the blanks with either true ($\checkmark$) or false ($\times$)
\begin{table}[ht]
 \centering
\begin{tabular}{|c|c|c|c|c|}
 \hline
  $f(n)$& $g(n)$& $f=O(g)$ & $f=\Omega (g)$& $f=\Theta(g)$ \\ \hline
  $2n^3+3n$& $100n^2+2n+100$& $\times$   & $\checkmark$ &  $\times$   \\ \hline
  $50n+\log n$& $10n+\log \log n$& $\checkmark$  &  $\checkmark$ &  $\checkmark$  \\ \hline
  $50n\log n$& $10n\log \log n$& $\times$   & $\checkmark$  &  $\times$ \\ \hline
  $\log n$& $ \log^2 n$&  $\checkmark$  & $\times$  & $\times$   \\ \hline
  $n!$& $ 5^n$&  $\times$  & $\checkmark$   &  $\times$  \\ \hline
 \end{tabular}
\end{table}
\end{hw}

\begin{hw}
\begin{enumerate}
\item Find two functions $f(x)$ and $g(x)$ such that $f(x)\neq O(g(x))$ and $g(x)\neq O(f(x))$.
\item Furthermore, we say a function $h:\mathbb{R}\rightarrow \mathbb{R}$ is \emph{monotonically increasing} if it satisfies the property `$x\leq y ~\Rightarrow~ h(x)\leq h(y)$'.
 \\
 Find two monotonically increasing functions $f(x)$ and $g(x)$ such that $f(x)\neq O(g(x))$ and $g(x)\neq O(f(x))$.
 \end{enumerate}
 \vspace{2mm}
    (Please give the detailed proof that your functions satisfy the requirements.)
\end{hw}
\begin{sol}
	\begin{enumerate}
		\item \label{q1} $f(x) = \cos x$ and $g(x) = \sin x$. If $f(x) = O(g(x))$, that is, $\cos x = O(\sin x)$. By the definition of $O$ notation, we have there exits a constant $x_0$ and a constant $C$, such that, for any $x \geq x_0$, there will be $|\cos x\ | \leq C \cdot \sin x$. Suppose we choose $x_1 \geq x_0$, and $\cos x_1 \neq 0$, since $|\cos x_1\ | \leq C \cdot \sin x_1$, when $x_2 = x_1 + \pi \geq x_0$, we have $|\cos x_1\ | \leq -C \cdot \sin x_1$. However this can not be true, since $\cos x_1 \neq 0$. Thus $f(x) \neq O(g(x))$. We can prove $g(x) \neq O(f(x))$ in a similiar way.
		
		\item $f(x) = e^{x+\sin x}$ and $g(x) = e^{x + \cos x}$. First, since $(x + \sin x)' = 1 + \cos x \geq 0$, thus $x + \sin x$ is monotonically increasing, then it would be easy to say $f(x)$ is also monotonically increasing. We can prove $g(x)$ is monotonically increasing in a similiar way. Second, if $f(x) = O(g(x))$, then by definition, we have there exits a constant $x_0$ and a constant $C$, such that, for any $x \geq x_0$, there will be $|e^{x+\sin x}| \leq C \cdot e^{x + \cos x}$, similiar as (\ref{q1}), we can prove it is not true, thus $f(x) \neq O(g(x))$, again we can prove $g(x) \neq O(f(x))$.
	\end{enumerate}
\end{sol}


\begin{hw}
 Prove that

\begin{enumerate}[(a)]
  \item $\left(1+ \frac{1}{n}\right)^n\leq e$ for all $n\geq 1$.
  \item $\left(1+\frac{1}{n}\right)^{n+1}\geq e$ for all $n\geq 1$.
  \item Using $(a)$ and $(b)$, conclude that $\lim_{n\rightarrow \infty}\left(1+\frac{1}{n}\right)^n =e$.
 % \item Prove $\left(1-\frac{1}{n}\right)^n\leq \frac{1}{e}\leq \left(1-\frac{1}{n}\right)^{n-1}$.
\end{enumerate}
\end{hw}
\begin{proof}
	\noindent
	\begin{enumerate}[(a)]
		\item A well known inequality is that $1 + x \leq e^{x}$, if we let $x = \frac{1}{n}$, we have $\left(1+ \frac{1}{n}\right)^{n}\leq \left(e^{1/n}\right)^{n} = e$.
		
		\item $\left(1+\frac{1}{n}\right)^{n+1} = \left(\frac{n+1}{n}\right)^{n+1} = \left(\frac{1}{1 - \frac{1}{n+1}}\right)^{n+1} \geq \left( e^{\frac{1}{n+1}}\right)^{n+1} = e$.
		
		\item 
		From $(a),(b)$ we can see that $\lim_{n \rightarrow +\infty}\left(1+ \frac{1}{n}\right)^n\leq e$ and $e \leq \lim_{n \rightarrow +\infty}\left(1+\frac{1}{n}\right)^{n+1}$. Since $$\lim_{n \rightarrow +\infty}\frac{\left(1+ \frac{1}{n}\right)^n}{\left(1+\frac{1}{n}\right)^{n+1}} = \lim_{n \rightarrow +\infty}\frac{1}{1 + \frac{1}{n}} = 1$$
		Thus $\lim_{n \rightarrow +\infty}\left(1+ \frac{1}{n}\right)^n\leq e \leq \lim_{n \rightarrow +\infty}\left(1+\frac{1}{n}\right)^{n+1} = \lim_{n \rightarrow +\infty}\left(1+ \frac{1}{n}\right)^n$, so $\lim_{n\rightarrow \infty}\left(1+\frac{1}{n}\right)^n =e$.
	\end{enumerate}
\end{proof}


\begin{hw}
Prove \emph{Bernoulli's inequality}: for each natural number $n$ and for every real $x\geq -1$, we have $(1+x)^n\geq 1+nx$.
\end{hw}
\begin{proof}
	We prove this statement by induction on $n$.\\
	\textbf{Basis step.} $n = 0$, for every real $x\geq -1$, we have $(1+x)^{0} = 1 \geq (1 + 0\times x) = 1$.\\
	\textbf{Induction hypothesis.} Assume when $n = k$, we have $(1+x)^{k} \geq 1 + kx$ for every $x \geq -1$.\\
	\textbf{Proof of induction step.} When $n = k+1$, since $x \geq -1$,we have $(1+x) \geq 0$ and $(1+x)^{k+1} = (1+x)(1+x)^{k} \geq (1+x)(1+kx) = 1 + (k+1)x + kx^{2} \geq 1 + (k+1)x$.\\
	Thus, we can say $(1+x)^n\geq 1+nx$ for each natural number $n$ and for every real $x\geq -1$.
\end{proof}

\begin{hw} Prove that for $n=1,2,\ldots,$ we have
\[
2\sqrt{n+1}-2<1+\frac{1}{\sqrt{2}}+\frac{1}{\sqrt{3}}+\cdots +\frac{1}{\sqrt{n}}\leq 2\sqrt{n}-1.
\]
\end{hw}
\begin{proof}
	We can prove this statement by induction on $n$.
	\begin{enumerate}
		\item \label{c1} We first prove $2\sqrt{n+1}-2<1+\frac{1}{\sqrt{2}}+\frac{1}{\sqrt{3}}+\cdots +\frac{1}{\sqrt{n}}$.\\
		\textbf{Basis step.} When $n = 1$, we have $2\sqrt{2} - 2 < 1$, which is true.\\
		\textbf{Induction hypothesis.} Assume when $n = k$ the statement is true, that is, $2\sqrt{k+1}-2<1+\frac{1}{\sqrt{2}}+\frac{1}{\sqrt{3}}+\cdots +\frac{1}{\sqrt{k}}$.\\
		\textbf{Proof of induction step.} When $n = k+1$, by induction hypothesis, we have $1+\frac{1}{\sqrt{2}}+\frac{1}{\sqrt{3}}+\cdots +\frac{1}{\sqrt{k}} + \frac{1}{\sqrt{k+1}} > 2\sqrt{k+1}-2 + \frac{1}{\sqrt{k+1}}$. Thus, we just need to prove that $2\sqrt{k+2}-2 < 2\sqrt{k+1}-2 + \frac{1}{\sqrt{k+1}}$, that is, $2\sqrt{k+2} < 2\sqrt{k+1} + \frac{1}{\sqrt{k+1}}$. Since 
		\begin{displaymath}
			\begin{tabular}{lcl}
				$2\sqrt{k+2} < 2\sqrt{k+1} + \frac{1}{\sqrt{k+1}}$ & $\Longleftrightarrow$ & $2\sqrt{(k+2)(k+1)} < 2(k+1) +1$ \\ 
				 & $\Longleftrightarrow$ & $4(k^2 + 3k +2) < 4k^2 + 12k + 9$ \\
				 & $\Longleftrightarrow$ & $8 < 9$
			\end{tabular}
		\end{displaymath}
		Thus, we can say $2\sqrt{n+1}-2<1+\frac{1}{\sqrt{2}}+\frac{1}{\sqrt{3}}+\cdots +\frac{1}{\sqrt{n}}$ is right.
		
		\item \label{c2} Then, we prove $1+\frac{1}{\sqrt{2}}+\frac{1}{\sqrt{3}}+\cdots +\frac{1}{\sqrt{n}}\leq 2\sqrt{n}-1$.\\
		\textbf{Basis step.} When $n = 1$, we have $1 \leq 2\sqrt{1}-1 = 1$, which is true.\\
		\textbf{Induction hypothesis.} Assume when $n = k$ the statement is true, that is, $1+\frac{1}{\sqrt{2}}+\frac{1}{\sqrt{3}}+\cdots +\frac{1}{\sqrt{k}} \leq 2\sqrt{k}-1$.\\
		\textbf{Proof of induction step.} When $n = k+1$, by induction hypothesis, we have $1+\frac{1}{\sqrt{2}}+\frac{1}{\sqrt{3}}+\cdots +\frac{1}{\sqrt{k}} + \frac{1}{\sqrt{k+1}} \leq 2\sqrt{k}-1 + \frac{1}{\sqrt{k+1}}$. Thus we just need to prove $2\sqrt{k}-1 + \frac{1}{\sqrt{k+1}} \leq 2\sqrt{k+1}-1$, that is, $2\sqrt{k} + \frac{1}{\sqrt{k+1}} \leq 2\sqrt{k+1}$. Since 
		\begin{displaymath}
		\begin{tabular}{lcl}
		$2\sqrt{k} + \frac{1}{\sqrt{k+1}} \leq 2\sqrt{k+1}$ & $\Longleftrightarrow$ & $2\sqrt{k(k+1)} + 1 \leq 2(k+1)$ \\ 
		& $\Longleftrightarrow$ & $4(k^2 + k) \leq 4k^2 + 4k + 1$ \\
		& $\Longleftrightarrow$ & $0 < 1$
		\end{tabular}
		\end{displaymath}
		Thus, we can say $1+\frac{1}{\sqrt{2}}+\frac{1}{\sqrt{3}}+\cdots +\frac{1}{\sqrt{n}}\leq 2\sqrt{n}-1$ is right.
	\end{enumerate}
	From \ref{c1} and \ref{c2}, we can say the statement is true.
\end{proof}


\begin{hw}
\hspace{1mm}
\begin{enumerate}[a)]
  \item Show that the product of all primes $p$ with $m<p\leq 2m$ is at most ${2m\choose m}$.
  \item Using a), prove the estimate $\pi(x)=\mathcal{O}(\frac{x}{\ln x})$, where $\pi(x)$ denote the number of primes not exceeding the number $x$.
%  \item Let $p$ be a prime, and let $m,k$ be natural numbers. Prove that if $p^k$ divides ${2m \choose m}$ then $p^k\leq 2m$.
%  \item Using c), prove $\pi(n)=\Omega(\frac{n}{\ln n})$.
\end{enumerate}

\end{hw}
\begin{sol}
	\begin{enumerate}[a)]
		\item \label{a} For all primes $p$ with $m < p \leq 2m$, we have $p\mid 2m!$, and we also have $p \nmid m!$, thus $p \nmid (m!)^2$. Since $2m! = {2m \choose m} \times (m!)^{2}$, so we have For all primes $p$ with $m < p \leq 2m$, we have $p\mid {2m \choose m}$, that is, the product of all primes $p$ with $m<p\leq 2m$ is at most ${2m\choose m}$.
		
		\item For any $x$, we say there will be a natural number $k$, such that $2^{k-1} < x \leq 2^{k}$. From \ref{a}) we know $\prod_{m < p \leq 2m}p \leq {2m \choose m}$, where p denotes the number is a prime. Thus, we have $\sum_{m < p \leq 2m}\ln p \leq \ln {2m \choose m}$. Since ${2m \choose m} \leq 2^{2m}$ and $\sum_{m < p \leq 2m}\ln p \geq  (\pi(2m) - \pi(m))\ln m$. So we now have $(\pi(2m) - \pi(m))\ln m \leq 2m\ln 2$, that is, $ \pi(2m) - \pi(m) \leq 2 \ln 2 \frac{m}{\ln m}$. If $m = 2^{h}$, we have $\pi(2^{h+1}) - \pi(2^{h}) \leq \frac{2^{h+1}}{h}$. It is easy to know $\pi(2^{h+1}) \leq 2^{h}$, thus we have $(h+1)\pi(2^{h+1}) - h\pi(2^{h}) \leq 3\cdot2^{h}$. So $\sum_{h = 0}^{k-1}((h+1)\pi(2^{h+1}) - h\pi(2^{h})) = k\pi(2^{k}) \leq \sum_{h = 0}^{k - 1}3\cdot2^{h} = 3\cdot2^{k}$. Thus, $\pi(2^{k}) \leq 3\cdot \frac{2^k}{k}$, since $2^{k-1} < x \leq 2^{k}$, we have $\pi(x) \leq \pi(2^k) \leq 3\cdot \frac{2^k}{k}$, so $\pi(x)=\mathcal{O}(\frac{x}{\ln x})$.
	\end{enumerate}
\end{sol}





\end{document}

%%% Local Variables:
%%% mode: tex-pdf
%%% TeX-master: t
%%% End:
